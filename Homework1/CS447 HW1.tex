
\documentclass[12pt]{article}\usepackage{amsmath}

\usepackage{fancyhdr}
\usepackage{lastpage}
\pagestyle{fancy}
\lhead{\footnotesize \parbox{11cm}{Joel Anna} }
%\lfoot{\footnotesize \parbox{11cm}{\textit{2}}}
\cfoot{}
\rhead{\footnotesize HW1:  \thepage\ of \pageref{LastPage}}
%\rfoot{\footnotesize Page \thepage\ of \pageref{LastPage}}
\renewcommand{\headheight}{24pt}
%\renewcommand{\footrulewidth}{0.4pt}

\begin{document}
\author{Joel Anna<annajoel@pdx.edu>}
\noindent
\underline{Question 1:}\\
a. How to find the direction vector \textbf{v} that points from \textbf{a} toward \textbf{b}.
Subtract \textbf{a} from \textbf{b}. $v = b - a$ \\
\textbf{v} = \textbf{[}$b_{\textit{x}}-a_{\textit{x}}$, $ b_{\textit{y}}-a_{\textit{y}}$\textbf{]}
\\\\
b. How the length of \textbf{v} computed.
The length of \textbf{v},$\|\textbf{v}\|$ can be computed by taking the square root of the dot product of \textbf{v} with itself.\\
$\|\textbf{v}\|$ = $\sqrt{\textbf{v} \cdot \textbf{v}}$ = $\sqrt{{v_x}^2 + {v_y}^2}$
\\\\
c. How to normalize \textbf{v}.\\
To normalize \textbf{v}, first compute $\|\textbf{v}\|$. Multiply \textbf{v} by the scalar $\frac{1}{\|\textbf{v}\|}$ to get \textbf{\^{v}}.\\
\textbf{\^{v}} = \textbf{[}$\frac{v_x}{\|\textbf{v}\|}, \frac{v_y}{\|\textbf{v}\|}$\textbf{]}
\\\\
\underline{Question 2:}\\
a. The dot product \textbf{a} $\cdot$ \textbf{b} is computed by taking the sum of the products of the components of \textbf{a} and \textbf{b}\\
 \textbf{a} $\cdot$ \textbf{b} = $a_{x}b_{x} + a_{y}b_{y} + a_{z}b_{z}$
 \\
 \\
 b. The relationship between \textbf{a} $\cdot$ \textbf{b} and the angle $\theta$ between \textbf{a} and \textbf{b}.\\
 \textbf{a} $\cdot$ \textbf{b} = $\|\textbf{a}\|\|\textbf{b}\|$cos $\theta$
 \\\\
 c. How the cross product vector \textbf{c} = \textbf{a} $\times$ \textbf{b} is computed.\\
$c_x = a_{y}b_{z} - a_{z}b_{y}$ \\
$c_y = a_{z}b_{x} - a_{x}b_{z}$ \\
$c_z = a_{x}b_{y} - a_{y}b_{x}$ \\
\\\\
d. The geometric relationship between a,b,c is.\\
if c is 0, a and b are parallel, otherwise c is orthogonal to a and b.
\\\\
e. The geometric relationship between \textbf{a} $\times$ \textbf{b} and \textbf{b} $\times$ \textbf{a}.\\
\textbf{a} $\times$ \textbf{b} = -(\textbf{b} $\times$ \textbf{a})
\\\\
f. The relationship between \textbf{a} $\times$ \textbf{b} and the angle $\theta$ between a and b.\\
 $\|\textbf{a} \times \textbf{b}\|$ = $\|\textbf{a}\|\|\textbf{b}\|$sin $\theta$
  \pagebreak
 \\
 \underline{Question 3:}\\
 $x^2 +3x +2 = 0$\\
 $x = \frac{-3 \pm \sqrt{9-8}}{2} \;=\; \frac{-3 \pm 1}{2} = -1, -2$\\
$ x = -1 ; \; x = -2$\\
\\
\underline{Question 4:}\\
 The distance from 2D point p to line $ax + by +c = 0$ is:\\
$\frac{| ap_x + bp_y + c |}{\sqrt{a^2 + b^2}}$\\
\\
\underline{Question 5:}\\
a. The minimum number of points to define a unique line in 3D that passes through all the points is 2. The points must not be co-linear.
\\\\
b. In general it is not possible to find one line that passes through more than the minimum number of points.
\\\\
c. For the 3D parametric line $\textbf{p} = o + t\textbf{d}$ the vectors \textbf{o} and \textbf{d} are defined by $p_1, p_2$ as:\\
\textbf{o} = 
$\begin{bmatrix}
p_{1x}\\
p_{1y}\\
p_{1z}
\end{bmatrix}$
\textbf{d} = 
$\begin{bmatrix}
p_{2x} - p_{1x}\\
p_{2y} - p_{1y}\\
p_{2z} - p_{1z}
\end{bmatrix}$
\\
\\\\
\underline{Question 6:}\\
Matrix multiply\\
$ \begin{bmatrix}
  1 & 2 & 5 \\
  4 & 1 & 12 \\
  3 & 1 & 5
 \end{bmatrix}$
 $ \begin{bmatrix}
   2 \\
   1 \\
   3
  \end{bmatrix}$=
$\begin{bmatrix}
1*2 \,+\, 2*1 \,+\,\,5*3 \\
4*2 \,+\, 1*1 \,+\,12*3 \\
3*2 \,+\, 1*1 \,+\,\, 5*3
\end{bmatrix}$=
$\begin{bmatrix}
 19 \\
 45 \\
 22
\end{bmatrix}$

\end{document}$\